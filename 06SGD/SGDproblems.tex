\documentclass{article}
\input ../preamble
\parindent = 0em

%\newcommand{\solution}[1]{}
\newcommand{\solution}[1]{\bigskip {\color{red} {\bf Solution}: #1}}

\begin{document}


\centerline{\bf TTIC 31230 Fundamentals of Deep Learning}
\bigskip
\centerline{\bf SGD Problems.}

\bigskip
\bigskip
~{\bf Problem 1. Reformulating Momentum as a Running Average.} Consider the following running update equation.

\begin{eqnarray*}
  y_0  & = & 0 \\
  y_t & = & \left(1 - \frac{1}{N}\right)y_{t-1} + x_t
\end{eqnarray*}

(a) Assume that $y_t$ converges to a limit, i.e., that $\lim_{t \rightarrow \infty} y_t$ exists.
If the input sequence is constant with $x_t = c$ for all $t \geq 1$, what is $\lim_{t \rightarrow \infty}\;y_t$?  Give a derivation of your answer
(Hint: you do not need to compute a closed form solution for $y_t$).

\solution{

  The limit $y_\infty$ must satisfy
  $$y_\infty = \left(1-\frac{1}{N}\right)y_\infty + c$$
  giving $y_\infty = Nc$.
}

\medskip
(b) $y_t$ is a running average of what quantity?

\solution{
  The update can be rewritten as
  $$y_t = \left(1 - \frac{1}{N}\right)y_{t-1} + \frac{1}{N}(Nx_t)$$
  so $y_t$ is the running average of $Nx_t$.
}

\medskip
(c) Express $y_t$ as a function of $\mu_t$ where $\mu_t$ is defined by

\begin{eqnarray*}
  \mu_0  & = & 0 \\
  \mu_t & = & \left(1 - \frac{1}{N}\right)\mu_{t-1} + \frac{1}{N}x_t
\end{eqnarray*}

\solution{
  $y_t$ is the running average of $Nx_t$ which equals $N$ times the running average of $x_t$ so we have
  $$y_t = N \mu_t$$
}

\bigskip
~{\bf Problem 2.  Bias Correction}
Consider the following update equation for computing $y_1,\ldots,y_t$ from $x_1,\ldots,x_t$.
\begin{eqnarray*}
  y_t & = & \left(1 - \frac{1}{\min(t,N)}\right)y_{t-1} + \frac{1}{\min(t,N)}\;x_t
\end{eqnarray*}

If $x_t = c$ for all $t \geq 1$ give a closed form solution for $y_t$.

\solution{
  For $t = 1$ we get $y_1 = x_1 = c$.  We then get that $y_{t+1}$ is a convex combination of $y_t$ and $x_t$ which maintains the invariant that $y_t = c$.
}


\bigskip
~{\bf Problem 3. Batch Size Coupling to RMSProp and Adam.}
Consider the following for-loop representation of a batch of matrix-vector products.

$$\mbox{for}\;b,i,j\;\;y[b,j] \;\pluseq\; W[j,i]x[b,i]$$

(a) Write the for-loop representation of back-propagation to $W.\grad$ following the convention that parameter gradients are averaged over the batch.

\solution{
  $$\mbox{for}\;b,i,j\;\;w.\grad[j,i] \pluseq \frac{1}{B}\;y.\grad[b,j]x[b,i]$$
}

\medskip
(b) Write a for-loop representation for computing $W.\grad[b,i,j]$ where this is the derivative of loss with respect to $W[i,j]$ for batch element $b$.

\solution{
  $$\mbox{for}\;b,i,j\;\;w.\grad[b,j,i] \pluseq \frac{1}{B}\;y.\grad[b,j]x[b,i]$$
}

\medskip
(c) Consider
$$W.\grad2[j,i] = \frac{1}{B} \;\sum_b\; W.\grad[b,j,i]^2$$
Is it possible to compute $W.\grad2[j,i]$ from $W.\grad[j,i]$? Explain your answer.

\solution{
  No. $W.\grad2[j,i]$ is the average over the batch of the of the square of the gradient.
  The average value does not determine the average square value --- the average value does not determine the variance.
}

\medskip
(d) Explain how your answer to (c) is related to batch size scaling of RMSProp and Adam.

\solution{
Adam and RMSProp both compute a running average of $\hat{g}[i]^2$ defined by
$$s_{t+1}[i] = \left(1-\frac{1}{N_s}\right)s_t + \frac{1}{N_s}\hat{g}[i]^2$$
At batch sized greater than 1 this fails to take into account the variance of the
gradiants within the batch.  This implies that $s_t[i]$ will be reduced as the batch size increases
and in the limit of large batches $s_t[i]$ will converge to the mean squared rather than the second moment.
}

\end{document}
